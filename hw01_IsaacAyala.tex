\documentclass[a4paper,12pt]{article}
%\documentclass[a4paper,12pt]{scrartcl}

\usepackage{xltxtra}

\input{../preamble.tex}

\usepackage[spanish]{babel}

% \setromanfont[Mapping=tex-text]{Linux Libertine O}
% \setsansfont[Mapping=tex-text]{DejaVu Sans}
% \setmonofont[Mapping=tex-text]{DejaVu Sans Mono}

\title{Selección de balancín}
\author{Isaac Ayala Lozano}
\date{2020-01-24}

\begin{document}
\maketitle

\section{Problema}

Seleccionar un balancín que cumpla con las especificaciones mostradas en la tabla \ref{tab: specs}.

\begin{table}[htb!]
 \begin{center}
\begin{tabular}{ll}
\hline
Carga & $> 100 \ \text{kgf}$ \\
Carrera & $100 \ cm$ \\
Peso neto & 20 a 30 kg\\
Presión de trabajo & 6 a 8 bar\\
\hline
\end{tabular}
\end{center}
\caption{Requisitos de operación.}
\label{tab: specs}
\end{table}

\section{Propuesta}

Se propone el uso del balancín \emph{ZAW035080} de Ingersoll Rand \cite{ingersollRand} para el manejo del material. 
Las especificaciones técnicas del balancín (tabla \ref{tab: balancer specs}) cumplen con los requisitos de operación solicitados.

El balancín se escoge de tipo neumático debido al ciclo de trabajo esperado. 
Un balancín eléctrico no soporta ciclos de trabajo tan intensos como un balancín neumático.
Se seleccionó un modelo capaz de levantar una carga 1.5 veces mayor a la carga nominal como medida de seguridad, teniendo así un factor de seguridad de 1.5.
La carrera máxima supera la distancia de trabajo por un factor de dos.


\begin{table}[htb!]
 \begin{center}
\begin{tabular}{ll}
\hline
Carga máxima (a 100 psi) & $158 \ \text{kgf}$ \\
Carrera máxima & $2032 \ \text{mm}$ \\
Peso neto & $28 \ \text{kg}$\\
Tracción & Cable sencillo\\
Botonera de control & Recta \\
Tipo de alimentación & Neumática (100 psi | 6.9 bar)\\
\hline
\end{tabular}
\end{center}
\caption{ZAW035080 - Especificaciones técnicas.}
\label{tab: balancer specs}
\end{table}


 \printbibliography

\end{document}
